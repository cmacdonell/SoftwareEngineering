\documentclass[letterpaper]{article}

\usepackage{url}
\usepackage[letterpaper,top=1in,bottom=1in,right=1in,left=1.2in]{geometry}
\usepackage{lastpage}
\usepackage{fancyhdr}
\pagestyle{fancy}
\usepackage{newcent}
\usepackage{color}
\usepackage{keystroke}
\usepackage{ifthen}
\usepackage{graphicx}

\newcommand{\marginpoints}[1]{\marginpar{\vspace{0.3cm}\hspace{1.5cm}\Huge\sfrac{}{#1}}}
\newcommand{\important}[0]{\marginpar{\hspace{2mm}\footnotesize \textcolor{red}{IMPORTANT}}}

\newcommand{\course}{CMPT 395}
\newcommand{\activity}{Lab \#3}
\newcommand{\assigned}{today}
\newcommand{\due}{today}
\newcommand{\duetime}{11:00am}
\newcommand{\weight}{2\% of final grade}

\newcommand{\horizrule}{\noindent\rule{\linewidth}{0.15mm}}

\lhead{\includegraphics[height=0.8cm]{../Images/macewan.jpeg}}
\chead{\course{}: \activity{}}
\rhead{Due \due{} at \duetime{}}
\cfoot{Page \thepage\ of \pageref{LastPage}}
\rfoot{\weight}

%\newenvironment{answer}{\begin{comment}}{\end{comment}}

\newenvironment{answer}
{
  \color{red}
}
{
  \vspace{3mm}
}

\title{\course{}: \activity{}}
\date{}

\begin{document}

\reversemarginpar

\begin{center}
  \Large{\activity{}: Version Control with {\tt git}}
\end{center}

\begin{center}
  \large{Objectives}
\end{center}

In this lab, you will learn about a version control system (VCS).
These systems help programmers maintain a repository of content, access historical versions, and over time, log a project's development.
One of the original such systems, the Source Code Control System (SCSS), started in the early 1970s~[1].
In this lab, we will use a modern system named {\tt git}: Linus Torvalds began developing it on April~3, 2005.
A couple of weeks later, on April~16, he created the first git repository with 6.7 million lines of code belonging to the Linux kernel.
\vspace{3mm}\\
\horizrule

\begin{center}
  \large{Instructions}
\end{center}

During the lab period, you must complete the tasks described below and present the result to your instructor.
To complete the task, you may work together with other students in the class and ask for help from your instructor.  {\bf This lab is worth 2\% of your grade}.
Labs not demonstrated by the due date/time will be subject to loss of 2\%, excessive ridicule
and eye-rolling by your instructor.   There are two questions to answer, you
may answer on paper or email your answers to your instructor.
\vspace{3mm}\\
\horizrule

\subsection*{Introduction}

You will access {\tt git} using its command-line interface.
In this section, you will learn how to access its help system.

\begin{enumerate}
\item Within a Terminal window, view {\tt git}'s most commonly used commands:\\
  \verb+$ git+
\item Look at the {\tt usage} line.
  Notice the small list of arguments that apply to {\tt git}.
  Also notice that {\tt [<args>]} follows {\tt <command>}; this means that commands will take arguments not listed here.
\item Read the description for each of the following commands: {\tt add}, {\tt checkout}, {\tt commit}, {\tt diff}, {\tt grep}, {\tt log}, {\tt rm}, and {\tt status}.
  You will use these commands during this lab.
\item Confirm that commands do in fact take other arguments.
  View the help for the {\tt add} command:\\
  \verb+$ git help add+\\
  As an alternative, try using the {\tt man} command:\\
  \verb+$ man git-add+\\
\end{enumerate}

\subsection*{Configuring {\tt git} and Forking a project }

Suppose you want to use {\tt git} to control source code of your project.  (Of
course, you have to use it, but I just wanted the lab to sound politely
suggestive).  The first thing is to configure your identity.

\begin{enumerate}
\item Configure this repository with your name and e-mail address:
  \begin{verbatim}
$ git config --global user.name "John A. Student"
$ git config --global user.email "student@mymail.macewan.ca"
\end{verbatim}

\item Configure {\tt git} for your favourite text editor.
  For Emacs, {\tt export GIT\_EDITOR=emacs} and for vim, {\tt export GIT\_EDITOR=vim}.

  If you are using {\tt git} on your own system, you could add this setting to one of your bash startup files.
  Think about which file you would use.
\item Configure {\tt git} to use colours for its display.
  Edit the file \verb+.gitconfig in your home directory+ to include the following lines:
  \begin{verbatim}
[color]
        branch = auto
        diff = auto
        status = auto
        ui = true
\end{verbatim}
\end{enumerate}

You will use this git repository for ALL your work in this course.

\subsection*{Start your task for this lab}

In this section, you will fork a repository and submit a ``Pull request''.  You must have configured git as above.  Any submissions from ``John Harvard'' will not be accepted

\begin{enumerate}
\item Login to github.com
\item Go the following link in a browser \verb+https://github.com/cmacdonell/395lab+
\item Fork this repository
\item Clone your forked copy of this repository from {\it your} github account not from Cam's account (use the https or ssh protocol in your clone command).
\end{enumerate}

Add your name and github account to the students.txt file.

\subsection*{Check your work into the repository}

In this section, you will save your work into the repository.
You can think of the repository as a database for files.

\begin{enumerate}
\item {\em Stage} files for inclusion in the repository ('add' is a confusing operation I know):
  \begin{verbatim}
git add students.txt
\end{verbatim}

  The idea of staging is an important concept.
  Whenever you actually add or change files in the repository (commit changes), you need to supply a description of the change.
  You wouldn't want to have to write such a description for every single file individually.
  By staging files, you are in essence tagging them for inclusion in the next commit.
\item View {\tt git}'s status:
  \begin{verbatim}
git status
\end{verbatim}
\item If {\tt git} clearly indicates that it is set to commit only the four identified files, {\em commit} your changes to the repository:
  \begin{verbatim}
git commit
\end{verbatim}
\item {\tt git} will display your text editor.
  If you want to abort the commit, simply exit the editor without saving.
  A description for a commit should be formatted as follows:
  \begin{verbatim}
Short one-line summary of the commit (less than 80 characters wide)

After leaving a blank line, you can then begin writing an optional
multi-line description of the commit.  You should also try to keep
these lines shorter than 80 characters.
\end{verbatim}
  For our example, we will enter a simple commit message (and save and close the editor):
  \begin{verbatim}
Added <my name> to the student list
\end{verbatim}
\item View the status again; you should no longer see the files listed.
\item View the log of the repository:
  \begin{verbatim}
git log
\end{verbatim}
\item Congratulations!
  You have made your first commit to your new repository.
\end{enumerate}

\subsection*{Working with a remote repository}

Git is handy even with only the above configuration.  It allows you to track
your work, find bugs and generally stay organized.

However, this is just the tip of the iceberg (how many times are we going to
use that metaphor? Seriously).  Git is an excellent collaboration tool because
all of the same properties apply when trying to figure out what other people
are working on within a project.

Now, please take the following oath:

\vspace{0.5cm}
{\large Whenever I am tempted to e-mail someone a tarball of code, I will stop
and use version control, possibly git.}
\vspace{0.5cm}

\subsubsection*{Back in your 395lab directory}

It is now time to push your changes to github.
Run the following:

\begin{verbatim}
$ git push origin
\end{verbatim}

{\bf Question 1: What are two reasons one might push code to a service
like github?}

\noindent
Answer:
\vspace{2in}

Now if you go to your github account you should be able to see your commit for your project.

\vspace{\baselineskip}
\noindent
{\bf Task: Issue a ``Pull Request'' to the account. Pull requests are the way your changes will likely be merged into a project.~[3]}

\section*{How git may save your life/sanity}

As a git user, the following three-step procedure should become
common and understandable for you:

\begin{verbatim}
$ git add <somefile>
$ git commit
$ git push origin
\end{verbatim}

What you are doing is:

\begin{enumerate}
\item Telling git to add the file to the repository.
\item Confirming the addition and giving a super-helpful commit message that
makes sense to other people, not just yourself.
\item Moving that commit to the remote copy of your code.
\end{enumerate}

If you are just using git locally you don't need to to do a \verb+git push+.
But for this course, you should push regularly.  {\bf You do not need to push
with every commit.  You should make commits very regularly and push when you
want to share your code}

You can also add multiple files at once.

\subsection*{Restoring a file}

\begin{enumerate}
\item Delete the students.txt in {\tt 395lab} that you just commited to.
\item View the status of the working directory.
\item Retrieve the file from the repository:
\begin{verbatim}
git checkout students.txt
\end{verbatim}
\item View the status again; confirm that the file now exists.
\item Say ``whooooooa''.
\end{enumerate}

Take a second and try to catch your breath realizing all the moments in your
previous coding where this ``restore'' feature would've been handy and saved
you hours of re-doing work.  There will be many moments like this where git's
feature make you smile.

\subsection*{Viewing a previous commit}

\begin{enumerate}
\item View the log and make note (copy) the commit identifier for our second commit.
  (It's the big hexadecimal number.)
\item View the changes made by that commit:\\
{\tt git show} $<$paste big number here$>$
\end{enumerate}

\section*{Finish your task for this lab}

To get the 2\% credit for this lab, you must demonstrate the following to your instructor.

\begin{enumerate}
\item your git ID is set on your work machine.
\item your github is account is setup.
\item used github to ``clone'' the ``forked'' 395Lab project from {\tt http://github.com/cmacdonell/395lab}.  Browse
around github, there are a lot of cool projects there.
\item Issue a pull request on github for your changes to Cam
\item Answer questions 1 and 2.
\end{enumerate}

{\bf Question 2: What is another VCS tool?  Find an online service that offers similar features to github that uses that tool}.

\noindent
Answer:
\vspace{2in}

\section*{Other good stuff}

\noindent
If you have time, I also suggest that you learn and play around with branching.

\begin{enumerate}
\item Branching {\tt http://learn.github.com/p/branching.html}
\end{enumerate}

\subsection*{References}

\begin{description} 
\item[{[1]}] J.~Loeliger. {\em Version Control with Git}. O'Reilly, Sebastopol, CA, May 2009.
\item[{[2]}] Forking on github {\tt https://help.github.com/articles/fork-a-repo}
\item[{[3]}] Github Pull Requests {\tt https://help.github.com/articles/using-pull-requests}
\end{description}

\end{document}

\documentclass[letterpaper]{article}

\usepackage{url}
\usepackage[letterpaper,top=1in,bottom=1in,right=1in,left=1.2in]{geometry}
\usepackage{lastpage}
\usepackage{fancyhdr}
\pagestyle{fancy}
\usepackage{newcent}
\usepackage{color}
\usepackage{keystroke}
\usepackage{ifthen}
\usepackage{parskip}
\usepackage{listings}
\usepackage{graphicx}

\setlength{\parindent}{0in}

\def\urltilde{\kern -.15em\lower .7ex\hbox{\~{}}\kern .04em}
\newcommand{\marginpoints}[1]{\marginpar{\vspace{0.3cm}\hspace{1.5cm}\Huge\sfrac{}{#1}}}
\newcommand{\important}[0]{\marginpar{\hspace{2mm}\footnotesize \textcolor{red}{IMPORTANT}}}

\newcommand{\superscript}[1]{\ensuremath{^{\textrm{#1}}}}
\newcommand{\course}{CMPT 395}
\newcommand{\activity}{Lab \#3}
\newcommand{\assigned}{today}
\newcommand{\due}{today}
\newcommand{\duetime}{2:00 PM}
\newcommand{\weight}{1\% of final grade}

\newcommand{\horizrule}{\noindent\rule{\linewidth}{0.15mm}}

\lstset{%
%  language=[LaTeX]TeX,     % latex breaks at underscores, by default it
%  doesn't
%  backgroundcolor=\color{gray!25},
  literate={\_}{}{0\discretionary{\_}{}{\_}},
  basicstyle=\footnotesize\ttfamily,
%  basicstyle=\ttfamily,
  numbersep=5pt,                  % how far the line-numbers are from the code
  breaklines=true,
%  breakatwhitespace=false,
  columns=fullflexible,
  xleftmargin=17pt  % indent the left margin so the numbers are within the textwidth
}


\lhead{\includegraphics[height=1cm]{../Images/macewan.jpeg}}
\chead{\course{}: \activity{}}
%\rhead{Due \due{} at \duetime{}}
\cfoot{Page \thepage\ of \pageref{LastPage}}

%\newenvironment{answer}{\begin{comment}}{\end{comment}}

\newenvironment{answer}
{
  \color{red}
}
{
  \vspace{3mm}
}

\title{\course{}: \activity{}}
\date{}

\begin{document}

\reversemarginpar

\vspace{0.5cm}

\begin{center}
  \Large{\activity{}: Ushahidi}
\end{center}

\vspace{0.5cm}

\begin{center}
  \large{Objectives}
\end{center}

In this lab you will deploy Ushahidi inside your VM.  Ushahidi is based upon
{\it Kohana} and follows the Model-View-Controller design.  The Ushahidi
framework is based upon Kohana and
understanding Kohana and MVC is key to working with Ushahidi.
\vspace{3mm}\\
\horizrule

\begin{center}
  \large{Prereqs}
\end{center}

You will need an account on github.com.  As per the instructions on the site,
please upload an SSH public key to the account.

\url{http://help.github.com/mac-set-up-git/}

If you aren't experienced with git, the git lab (on blackboard) is a good place
to start.

Two important steps are to set up your git identity

\begin{verbatim}
$ git config --global user.name "John A. Student"
$ git config --global user.email "student@mymail.macewan.ca"
\end{verbatim}

\begin{center}
  \large{Instructions}
\end{center}

% For Kohana, the first step is to adjust the site_domain, we may follow Kohana
% 101
%

\subsection*{Introduction}

Go to Ushahidi.com and Download version 2.1 (codenamed "Tunis").  Install it
into your {\tt public\_html} directory.  The instructions for installing here
are

\url{http://wiki.ushahidi.com/doku.php?id=how\_to\_install\_ushahidi}

You will need to create a database using PHPmyadmin in your VM

\url{http://192.168.56.50/phpmyadmin}

NOTE: your database must begin with ``jharvard\_'' or else the database will not
be created.

The VM is configured with most of the system requirements, you can jump ahead
to the ``Web Install''

You can follow the ``Basic Install''.  or if you really 

{\bf Very Important: You must to "Enable Clean URLs'' to NO on the 2nd page
of the basic install.}

Once the install completes, you can point your browser at

{\tt http://192.168.56.50/\urltilde jharvard/Ushahidi\_Web/} and you should see
your new Ushahidi installation.  If you encounter problems ask questions on the
discussion board or in class/lab.

You can now follow the assignment.  You may get help on importing the Irene or
Haiti data into your database.

Discussing the assignment tasks is covered by the academic dishonesty
guidelines.  Your work must be your own.

%\subsection*{References}

%\begin{description}
%\item[{[1]}] J.~Loeliger. {\em Version Control with Git}. O'Reilly, Sebastopol, CA, May 2009.
%\end{description}

\end{document}

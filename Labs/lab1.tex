\documentclass[letterpaper]{article}

\usepackage{url}
\usepackage[letterpaper,top=1in,bottom=1in,right=1in,left=1.2in]{geometry}
\usepackage{lastpage}
\usepackage{fancyhdr}
\pagestyle{fancy}
\usepackage{newcent}
\usepackage{listings}
\usepackage{color}
\usepackage{keystroke}
\usepackage{ifthen}
\usepackage{graphicx}

\newcommand{\task}[1]{\vspace{\baselineskip}\noindent{\bf #1}\vspace{\baselineskip}}
\newcommand{\marginpoints}[1]{\marginpar{\vspace{0.3cm}\hspace{1.5cm}\Huge\sfrac{}{#1}}}
\newcommand{\important}[0]{\marginpar{\hspace{2mm}\footnotesize \textcolor{red}{IMPORTANT}}}

\newcommand{\course}{CMPT 395}
\newcommand{\activity}{Lab \#1}
\newcommand{\assigned}{today}
\newcommand{\due}{}
\newcommand{\duetime}{11:59 PM}
%\newcommand{\weight}{10\% of final grade}

\newcommand{\horizrule}{\noindent\rule{\linewidth}{0.15mm}}

\lhead{\includegraphics[height=0.8cm]{../Images/macewan.jpeg}}
\chead{\course{}: \activity{}}
\rhead{Due \due{} at \duetime{}}
\cfoot{Page \thepage\ of \pageref{LastPage}}

%\newenvironment{answer}{\begin{comment}}{\end{comment}}
\lstset{%
%  language=C,     % latex breaks at underscores, by default it
%  doesn't
%  backgroundcolor=\color{gray!25},
  literate={\_}{}{0\discretionary{\_}{}{\_}},
  basicstyle=\ttfamily,
%  basicstyle=\ttfamily,
%  numbers=left,
  numberstyle=\small\color{gray},
  %numbersep=5pt,                  % how far the line-numbers are from the code
  breaklines=true,
%  breakatwhitespace=false,
  columns=fullflexible,
  xleftmargin=17pt  % indent the left margin so the numbers are within the textwidth
}


\newenvironment{answer}
{
  \color{red}
}
{
  \vspace{3mm}
}

\title{\course{}: \activity{}}
\date{}

\begin{document}

\reversemarginpar

\begin{center}
  \Large{\activity{}: Running code in a VM}
\end{center}

\begin{center}
  \large{Objectives}

\begin{enumerate}
\item Install and configure a VM Appliance
\item Play around with PHP
\item Start forming project groups
\end{enumerate}
\end{center}

%\vspace{3mm}\\
%\horizrule

\begin{center}
  \large{Instructions}
\end{center}

%%
%%
%%You will access {\tt git} using its command-line interface.
%%In this section, you will learn how to access its help system.
%%
%%\begin{enumerate}
%%\item Within a Terminal window, view {\tt git}'s most commonly used commands:\\
%%  \verb+$ git+
%%\item Look at the {\tt usage} line.
%%  Notice the small list of arguments that apply to {\tt git}.
%%  Also notice that {\tt [<args>]} follows {\tt <command>}; this means that commands will take arguments not listed here.
%%\item Read the description for each of the following commands: {\tt add}, {\tt checkout}, {\tt commit}, {\tt diff}, {\tt grep}, {\tt init}, {\tt log}, {\tt rm}, and {\tt status}.
%%  You will use these commands during this lab.
%%\item Confirm that commands do in fact take unlisted arguments.
%%  View the help for the {\tt init} command:\\
%%  \verb+$ git help init+\\
%%  As an alternative, try using the {\tt man} command:\\
%%  \verb+$ man git-init+\\
%%\end{enumerate}
%%

\subsection*{Introduction}

In this lab, you will be
familiarizing yourelf with working within a
virtual machine environment as well as some of the software involved in this course.  Virtual machines are a great way to provide
consistency and to not muddle your primary OS with a lot of installation files.

For all your work, you should be making regular backups.  This is also true of
your virtual machine.  Whether you are working on a lab machine or laptop, you
should regularly make a copy of your VM to separate disk or flash drive.

%%\begin{center}
%%\framebox[\linewidth]{
%%\parbox[c]{0.9\linewidth}{
%%\center{\bf Standard Comment About Design Decisions}
%%\vspace{\baselineskip}
%%
%%Although many details about this assignment are given in this description,
%%there are many other design decisions that are left for you to make. In those
%%cases, you should make reasonable design decisions (e.g., that do not
%%contradict what we have said and do not significantly change the purpose of the
%%assignment), document them in your source code, and discuss them in your
%%report. Of course, you may ask questions about this assignment (for example, on
%%blackboard) and we may choose to provide more information or provide some
%%clarification. However, the basic requirements of this assignment will not
%%change.
%%}
%%}
%%\end{center}

\subsection*{Task 1: Install a Virtual Machine}

Virtual Machines are invaluable tools for testing and deploying software.  For
this course, we'll be using a VM from Harvard University's CS50 course
(\lstinline{www.cs50.net}).  The CS50 VM works easily with the VirtualBox VM
software which is free.  VirtualBox is installed on the lab machines or you can
install it on your laptop or home computer.

Get a copy of the VM from a flash drive from your instructor and {\bf import}
it into your VM.  Look for ``Import Appliance'' under the ``File'' menu.  Make
sure to use Version 3 of the CS50 VM!

The CS50 VM is loaded with lots of software.  Take some time to look around in
the VM.

\subsection*{Task 2: Run some PHP}

For your test code, I suggest you use jharvard's web directory.  To use a user web
directory, it must be named \lstinline{public_html} within jharvard's user
directory.  The permissions on the directory (and its parents) must be readable
and executable to allow the web server to access them.

Create a file under the public\_html directory named \lstinline{index.php} that contains the following:

\begin{lstlisting}
<?php
phpinfo();
?>
\end{lstlisting}

When you point your web browser at
\lstinline{http://192.168.56.101/~jharvard/} you should see a lot of purple
talking about PHP.  View the HTML page source.  How did so little PHP generate
so much data?

Now add the following code above the PHP code:

\begin{lstlisting}
<script type="text/javascript">
function isprime(num) {

    for (var i = 2; i < num/2; i++) {
        if (num % i == 0) {return false;}
    }
    return true;

}

function start() {
    document.write("<p align=center>Hello<br>");
    hello = 10;
    for (var i = 1; i < hello; i++) {
       
        if (isprime(i))
            document.write(i + " is prime<br>");
        else document.write(i + " is not prime<br>");
    }
    document.write("<p>");
}    
start();
</script>
\end{lstlisting}

Now reload the page.  Where did the new content come from?  View the page source again.

Having Python create web content is slightly more involved.  If you finish these other parts quickly, come see Cam.

\subsection*{Task 3: Observe the Ushahidi API stream}

There are 3000 deployments of Ushahidi in the U.S. alone.  One use it to map crime in Atlanta.  Check out \lstinline{http://crime.mapatl.com}.

Ushahidi has an Application Programming Interface (API) that allows
applications to modify and consume data about
the incident that Ushahidi is monitoring.

You can read about the API in the Ushahidi documentation (see Resources below).
For example, if you connect to the following URL:

\url{http://crime.mapatl.com/api?task=incidents&resp=xml}

You will get XML output of incidents that look something like this:

\begin{footnotesize}
\begin{verbatim}
<response>
  <payload>
    <domain>http://crime.mapatl.com/</domain>
    <incidents>
      <incident>
      <id>141477</id>
      <title>Larceny @ 650 W CONWAY DR NW</title>
      <description>Larceny @ 650 W CONWAY DR NW</description>
      <date>2012-12-15 23:00:00</date>
    .
    .
    .
  </payload>
</response>
\end{verbatim}
\end{footnotesize}

With the following url:

\url{http://crime.mapatl.com/api?task=incidents&resp=json}

You will get JSON output (which looks way messier) that contains the same information.

You can access other GET methods from the API other than "incidents" (i.e. version,
categories, countries, etc) in the same manner.

Understand that Ushahidi can present information in a number of different ways.

\subsection*{Task 4: Start thinking about the project you'd like to work on.}

You will have a choice of working on Ushahidi or iPython this term.  Based on
what you've seen, start talking with others and forming groups.  If you can't
find a group, please come see me.

%%
%%\subsection*{Task 4: YARRR! [3 marks]}
%%
%%The Ushahidi API stream that you modified in Task 3 takes a parameter called
%%{\tt resp}, it can be either {\tt resp=xml} or {\tt resp=json}.  It will
%%actually spit out XML if you type in anything other {\tt json} after the equal
%%sign in {\tt resp=}.  For this task, add a third response type named {\tt pirate},
%%that will not print XML or JSON, but will simply print {\tt YARRRR!} in the
%%browser (no HTML necessary).
%%


\subsection*{Resources}

\begin{footnotesize}
\begin{description}
\item[{[1]}] {\bf Google is your friend}
\item[{[2]}] {\bf https://manual.cs50.net/appliance/3/}
\end{description}
\end{footnotesize}

\end{document}

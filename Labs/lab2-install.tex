\documentclass[letterpaper]{article}

\usepackage{url}
\usepackage[letterpaper,top=1in,bottom=1in,right=1in,left=1.2in]{geometry}
\usepackage{lastpage}
\usepackage{fancyhdr}
\pagestyle{fancy}
\usepackage{newcent}
\usepackage{color}
\usepackage{keystroke}
\usepackage{ifthen}
\usepackage{parskip}
\usepackage{listings}
\usepackage{graphicx}

\setlength{\parindent}{0in}

\def\urltilde{\kern -.15em\lower .7ex\hbox{\~{}}\kern .04em}
\newcommand{\marginpoints}[1]{\marginpar{\vspace{0.3cm}\hspace{1.5cm}\Huge\sfrac{}{#1}}}
\newcommand{\important}[0]{\marginpar{\hspace{2mm}\footnotesize \textcolor{red}{IMPORTANT}}}

\newcommand{\superscript}[1]{\ensuremath{^{\textrm{#1}}}}
\newcommand{\course}{CMPT 395}
\newcommand{\activity}{Lab \#2}
\newcommand{\assigned}{today}
\newcommand{\due}{today}
\newcommand{\duetime}{2:00 PM}
\newcommand{\weight}{1\% of final grade}

\newcommand{\horizrule}{\noindent\rule{\linewidth}{0.15mm}}

\lstset{%
%  language=[LaTeX]TeX,     % latex breaks at underscores, by default it
%  doesn't
%  backgroundcolor=\color{gray!25},
  literate={\_}{}{0\discretionary{\_}{}{\_}},
  basicstyle=\footnotesize\ttfamily,
%  basicstyle=\ttfamily,
  numbersep=5pt,                  % how far the line-numbers are from the code
  breaklines=true,
%  breakatwhitespace=false,
  columns=fullflexible,
  xleftmargin=17pt  % indent the left margin so the numbers are within the textwidth
}


\lhead{\includegraphics[height=1cm]{../Images/macewan.jpeg}}
\chead{\course{}: \activity{}}
%\rhead{Due \due{} at \duetime{}}
\cfoot{Page \thepage\ of \pageref{LastPage}}

%\newenvironment{answer}{\begin{comment}}{\end{comment}}

\newenvironment{answer}
{
  \color{red}
}
{
  \vspace{3mm}
}

\title{\course{}: \activity{}}
\date{}

\begin{document}

\reversemarginpar

\vspace{0.5cm}

\begin{center}
  \Large{\activity{}: Installfest}
\end{center}

\vspace{0.5cm}

\begin{center}
  \large{Objectives}
\end{center}

In this lab you will setup your deployment of Ushahidi and IPython and get
some partners if you don't have any already.  This is the first step along the road
of working with open-source software and the general tasks you do here will help you
install open-source software in the future.
\vspace{3mm}\\
\horizrule

\begin{center}
  \large{Prereqs}
\end{center}

You will need an account on github.com.  As per the instructions on the site,
please upload an SSH public key to the account.

\url{http://help.github.com/mac-set-up-git/}

If you aren't experienced with git, the git lab (on blackboard) is a good place
to start.

Two important steps are to set up your git identity

\begin{verbatim}
$ git config --global user.name "John A. Student"
$ git config --global user.email "student@mymail.macewan.ca"
\end{verbatim}

\begin{center}
  \large{Instructions}
\end{center}

For this lab, the goal is to get your software up and running so that you can
get going with your project.  You may work together with your project partners.
If you do not yet have a group, please form one today.

Install one of the following, or if you have time, both.

Before starting, please read the project document, so you understand the goal
of the project.

\subsection*{Install Ushahidi}
o").  Install it
into your {\tt public\_html} directory.  The instructions for installing here
are available on the web.  Use your google skills to find them or search on the project home page.

Make note of any mistakes in the installation documents.  

You will need to create a database using PHPmyadmin in your VM

\url{http://192.168.56.50/phpmyadmin}

NOTE: your database must begin with ``jharvard\_'' or else the database will not
be created.

The VM is configured with most of the system requirements, you can jump ahead
to the ``Web Install''

You can follow the ``Basic Install''.  or if you really 

{\bf Very Important: You must to "Enable Clean URLs'' to NO on the 2nd page
of the basic install.}

Once the install completes, you can point your browser at

{\tt http://192.168.56.50/\urltilde jharvard/Ushahidi\_Web/} and you should see
your new Ushahidi installation.  If you encounter problems, ask questions. 

\subsection*{IPython}

If you are working on IPython then find IPython and install the 2.0.0-dev
version of IPython.  IPython is dependent on some additional packages that
will also need to be installed.

IPython does not install easily on the version of the CS50 VM we installed
last week.  I suggest that you either work on your laptop without a VM, or
install a more up-to-date version of Linux or work on the lab computers.  

Again, three things to do to help you move along

\begin{itemize}
\item Read the project description before starting
\item Work with your project mates
\item Make note of any mistakes or unclear parts of the IPython install documentation.
\end{itemize}

%\begin{description}
%\item[{[1]}] J.~Loeliger. {\em Version Control with Git}. O'Reilly, Sebastopol, CA, May 2009.
%\end{description}

\end{document}

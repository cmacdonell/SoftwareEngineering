\documentclass[letterpaper]{article}

\usepackage{url}
\usepackage[letterpaper,top=1in,bottom=1in,right=1in,left=1.2in]{geometry}
\usepackage{lastpage}
\usepackage{fancyhdr}
\pagestyle{fancy}
\usepackage{newcent}
\usepackage{color}
\usepackage{keystroke}
\usepackage{ifthen}
\usepackage{graphicx}
\DeclareGraphicsExtensions{.eps}

\newcommand{\marginpoints}[1]{\marginpar{\vspace{0.3cm}\hspace{1.5cm}\Huge\sfrac{}{#1}}}
\newcommand{\important}[0]{\marginpar{\hspace{2mm}\footnotesize \textcolor{red}{IMPORTANT}}}

\newcommand{\course}{CMPT 395}
\newcommand{\activity}{Lab \#1}
\newcommand{\assigned}{today}
\newcommand{\due}{today}
\newcommand{\duetime}{2:00 PM}
%\newcommand{\weight}{1\% of final grade}

\newcommand{\horizrule}{\noindent\rule{\linewidth}{0.15mm}}

\lhead{\includegraphics[height=0.5cm]{../Images/macewan.jpeg}}
\chead{\course{}: \activity{}}
%\rhead{Due \due{} at \duetime{}}
\cfoot{Page \thepage\ of \pageref{LastPage}}

%\newenvironment{answer}{\begin{comment}}{\end{comment}}

\newenvironment{answer}
{
  \color{red}
}
{
  \vspace{3mm}
}

\title{\course{}: \activity{}}
\date{}

\begin{document}

\reversemarginpar

\begin{center}
  \Large{\activity{}: Virtual Box and the CS50 VM}
\end{center}

\begin{center}
  \large{Objectives}
\end{center}

In this lab we will learn about our development environment for this course,
namely a virtual machine (VM).  Our VM is taken from Harvard's Introductory
course CS 50 (http://cs50.net) and uses the free VirtualBox software to run the
VM.  VirtualBox is installed in the labs and you may install it on your labtop
or home computer if you wish.
\vspace{3mm}\\
\horizrule

\begin{center}
  \large{Instructions}
\end{center}

During the lab period, you should complete the tasks described below and
present the result to your instructor.  To complete the task, you may work
together with other students in the class and ask for help from your
instructor.  Labs are not graded in this course.
\vspace{3mm}\\
\horizrule

\subsection*{Introduction}

VMs are an excellent tool for using operating systems and applications that are
either not able to be installed on your native OS, or have significant
dependencies (other applications, libraries, etc.)  We will be using a VM for
projects in this course since our development environment has many pieces of
software that can be difficult to configure.  As well, it will allow you to
work outside of the lab more easily if you wish.

\begin{enumerate}
\item Install VirtualBox
\item Add a Host-only network
  Under a configuration menu, there will be a section for the Network and it
will allow you to add a host-only network.  Use the default name for this new
network which should be {\tt vboxnet0}.
\item Download the image from this site: \\
  \verb+ https://manual.cs50.net/CS50_Appliance_2.3 +
\item Follow the installation instructions on the above site to import the
image into your installation of VirtualBox.  In particular,
Sections 2.3.10 and 2.3.15 are very useful.
\end{enumerate}


Following the installation, you should be able to get the VM running and
connect the web server at {\tt http://192.168.56.50}.

The CS50 VM is based upon Fedora Linux (www.fedoraproject.org) and
contains a lot of software that is useful for this course as well as
software development in general.

For this course, the VM contains: Apache, MySQL, PHP5

Other interesting software includes: DropBox integration, Scratch, Eclipse, gcc.

NOTE: the necessary software for this course (Apache, MySQL, PHP) runs on
Windows as well, so if you want to install on Windows, that choice is available
to you.  You will still be able to do the project assignments/sprints on
Windows if you wish to.

\subsection*{Updating and installing software with {\tt yum}}

Yum is the name of the package manager in Fedora Linux.  Yum can be used to
install and update software within the VM.

\verb+ $ yum search <package name> +

Installing software requires using `sudo' privileges

\begin{verbatim}
$ sudo yum install <package name>
\end{verbatim}

Let's try installing something.  In particular the package {\tt ctags} is
useful for navigating source code.  You can confirm it's available by running:

\begin{verbatim}
$ yum search ctags-etags
\end{verbatim}

Let's install it using 'sudo' to give the proper permissions to install
software.

\begin{verbatim}
$ sudo yum install ctags-etags
\end{verbatim}

\subsection*{Apache and PHP}

Apache is the name of the webserver that is installed in the VM image.  If you
point a web browser (either in the VM or on your laptop/desktop) at

\url{http://192.168.56.50}

you will see a page that is returned by Apache by default.

In jharvard's home directory (/home/jharvard) create a person web page directory.  This directory {\bf
must} be named {\tt public\_html}.  Running the following in the home directory
will help.

\verb+ $ mkdir public\_html +

We must give the web browser permission to access our public\_html directory:

\begin{verbatim}
$ chmod 755 .
$ chmod 755 public\_html
\end{verbatim}

Inside this directory, use your preferred editor to create a file named {\tt
index.php} and put the following code in it.

\begin{verbatim}
<?php
    phpinfo();
?>
\end{verbatim}

We must also give permission for the file:

\begin{verbatim}
$ chmod 644 index.php
\end{verbatim}

Now if you point a browser at

\url{http://192.168.56.50/~jharvard/index.php}

You should see a long page about the PHP installation on your system.

\subsection*{Playing with PHP}

Now, let's play with PHP a little.

PHP code is used to generate front-end browser code (i.e., Javascript, HTML,
etc).   PHP code can be interlaced with regular HTML code.

PHP files should be named have an extension of {\tt .php}.  The {\tt index.php}
example is an example of a very simple PHP script.

If you view the source of the page in your browser however, you will see that
the source of the page is very different.  The code is different because PHP is
a {\it server-side} script that generates a web front-end language, typically HTML.
The browser never sees the PHP code, the web server (Apache) runs the PHP,
generates the HTML and then sends the HTML to the web browser.

\subsubsection*{First modifications}

Change the line {\tt phpinfo();} in {\tt index.php} to {\tt echo 'hello';}

Notice the output in the browser (after reloading) will read 'hello' (you can
view the source too).

Now remove the first and third lines leaving only {\tt echo 'hello';} in the
file then reload the browser.

The text now shows the original contents of the file because we removed the
PHP directive {\tt <?php ... ?>} so the file is not interpreted as PHP by the
web server (even with a {\tt .php} extension), it is just sent to the web
browser as plain text.

%\subsubsection*{Lab challenge: FizzBuzz}

\end{document}

\documentclass[letterpaper]{article}

\usepackage{url}
\usepackage[letterpaper,top=1in,bottom=1in,right=1in,left=1.2in]{geometry}
\usepackage{lastpage}
\usepackage{fancyhdr}
\pagestyle{fancy}
\usepackage{newcent}
\usepackage{listings}
\usepackage{color}
\usepackage{keystroke}
\usepackage{ifthen}
\usepackage{graphicx}

\newcommand{\task}[1]{\vspace{\baselineskip}\noindent{\bf #1}\vspace{\baselineskip}}
\newcommand{\marginpoints}[1]{\marginpar{\vspace{0.3cm}\hspace{1.5cm}\Huge\sfrac{}{#1}}}
\newcommand{\important}[0]{\marginpar{\hspace{2mm}\footnotesize \textcolor{red}{IMPORTANT}}}

\newcommand{\course}{CMPT 395}
\newcommand{\activity}{Assignment \#2}
\newcommand{\assigned}{today}
\newcommand{\due}{January 24th}
\newcommand{\duetime}{11:59 PM}
\newcommand{\weight}{5\% of final grade}

\newcommand{\horizrule}{\noindent\rule{\linewidth}{0.15mm}}

\lhead{\includegraphics[height=1cm]{../Images/macewan.jpeg}}
\chead{\course{}: \activity{}}
%\rhead{Due \due{} at \duetime{}}
\cfoot{Page \thepage\ of \pageref{LastPage}}
\lfoot{\weight}

%\newenvironment{answer}{\begin{comment}}{\end{comment}}
\lstset{%
%  language=C,     % latex breaks at underscores, by default it
%  doesn't
%  backgroundcolor=\color{gray!25},
  literate={\_}{}{0\discretionary{\_}{}{\_}},
  basicstyle=\ttfamily,
%  basicstyle=\ttfamily,
%  numbers=left,
  numberstyle=\small\color{gray},
  %numbersep=5pt,                  % how far the line-numbers are from the code
  breaklines=true,
%  breakatwhitespace=false,
  columns=fullflexible,
  xleftmargin=17pt  % indent the left margin so the numbers are within the textwidth
}


\newenvironment{answer}
{
  \color{red}
}
{
  \vspace{3mm}
}

\title{\course{}: \activity{}}
\date{}

\begin{document}

\reversemarginpar

\begin{center}
  \Large{\activity{}: Article Comparison}
\end{center}

\begin{center}
  \large{Objectives}

\begin{enumerate}
\item Find {\bf two} online resources related to a topic from our class
\item Compare them to what we have discussed/practiced in class
\end{enumerate}
\end{center}

%\vspace{3mm}\\
%\horizrule

\begin{center}
  \large{Instructions}
\end{center}

%%
%%
%%You will access {\tt git} using its command-line interface.
%%In this section, you will learn how to access its help system.
%%
%%\begin{enumerate}
%%\item Within a Terminal window, view {\tt git}'s most commonly used commands:\\
%%  \verb+$ git+
%%\item Look at the {\tt usage} line.
%%  Notice the small list of arguments that apply to {\tt git}.
%%  Also notice that {\tt [<args>]} follows {\tt <command>}; this means that commands will take arguments not listed here.
%%\item Read the description for each of the following commands: {\tt add}, {\tt checkout}, {\tt commit}, {\tt diff}, {\tt grep}, {\tt init}, {\tt log}, {\tt rm}, and {\tt status}.
%%  You will use these commands during this lab.
%%\item Confirm that commands do in fact take unlisted arguments.
%%  View the help for the {\tt init} command:\\
%%  \verb+$ git help init+\\
%%  As an alternative, try using the {\tt man} command:\\
%%  \verb+$ man git-init+\\
%%\end{enumerate}
%%

\subsection*{Introduction}

In this assignment, you will search for blog posts and articles related to some topic we have discussed in class.  You must find two of them.  Your two \emph{references} that you find must discuss at reasonable length (no 1 paragraph posts), a topic we have learned about (Processes, Agile, Testing, Version Control, Open Source, etc).

Your two references \textbf{must differ in opinion} at some point.  For example, find an article that talks about the advantages of Scrum and find a blog post that criticizes Scrum.  Alternatively, find an article that states how wonderful git is, and find an article that takes a differing opinion.

The articles need not be polar opposites, but there must be some element of them that differs.

\subsection*{Composition}

There are three main aspects to this document that you must discuss at reasonable length (i.e. at least two paragraphs).  Your report should have the following sections.

\begin{enumerate}
\item Summary of first reference
\item Summary of second reference
\item Discussion of difference of opinion between the references	
\end{enumerate}

For each of the above aspects, you should relate to what was learned in class on your topic of choice.
In highlighting the differences between the article give an example of the particular concept that the articles differ on.

\subsection*{Assignment Deliverables}

Your document should be submitted as a single PDF of no more than 3 pages, but closer to 2 pages.  
Your document should be well-written, free of spelling and grammatical errors,
divided into sections with headings.  It must contain your name and student ID.

\subsection*{Submission}

Your final document must be emailed by April. 12 at 11:59pm to your instructor.  Again, send PDFs, not Word files or any
other format.

\subsection*{Marking}

Here is a breakdown of marks:

\begin{itemize}
\item Explanation and comparison of references - 60\%
\item Choice of references - 20\%
\item Document style and clarity - 20\%
\end{itemize}

%\subsection*{Resources}

%\begin{footnotesize}
%\begin{description}
%\end{description}
%\end{footnotesize}

\end{document}

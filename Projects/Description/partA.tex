\documentclass[letterpaper]{article}

\usepackage{url}
\usepackage[letterpaper,top=1in,bottom=1in,right=1in,left=1.2in]{geometry}
\usepackage{lastpage}
\usepackage{fancyhdr}
\pagestyle{fancy}
\usepackage{newcent}
\usepackage{color}
\usepackage{keystroke}
\usepackage{ifthen}
\usepackage{graphicx}
\usepackage{marvosym}

\newcommand{\marginpoints}[1]{\marginpar{\vspace{0.3cm}\hspace{1.5cm}\Huge\sfrac{}{#1}}}
\newcommand{\important}[0]{\marginpar{\hspace{2mm}\footnotesize \textcolor{red}{IMPORTANT}}}

\newcommand{\course}{CMPT 395}
\newcommand{\activity}{Project: Part A}
\newcommand{\assigned}{today}
\newcommand{\due}{}
\newcommand{\duetime}{}
\newcommand{\weight}{25\% of project}

\newcommand{\horizrule}{\noindent\rule{\linewidth}{0.15mm}}

\lhead{\includegraphics[height=0.8cm]{../../Images/macewan.jpeg}}
\chead{\course{} \activity{}}
\rhead{Weight: \weight{}}
%\rhead{Due \due{} at \duetime{}}
\cfoot{Page \thepage\ of \pageref{LastPage}}

%\newenvironment{answer}{\begin{comment}}{\end{comment}}

\newenvironment{answer}
{
  \color{red}
}
{
  \vspace{3mm}
}

\title{\course{}: \activity{}}
\date{}

\begin{document}

\reversemarginpar

\begin{center}
  \Large{\activity{}}
\end{center}

\begin{center}
  \large{Objectives}
\end{center}

As you know the goal of the project is to work with and potentially contribute
to an free and open-source software (FOSS) project.  The goal of the first
part, Part A, is really get
into that project, in short, to get ``Productively Lost''.

\horizrule

\begin{center}
  \large{Instructions}
\end{center}

%%
%%
%%You will access {\tt git} using its command-line interface.
%%In this section, you will learn how to access its help system.
%%
%%\begin{enumerate}
%%\item Within a Terminal window, view {\tt git}'s most commonly used commands:\\
%%  \verb+$ git+
%%\item Look at the {\tt usage} line.
%%  Notice the small list of arguments that apply to {\tt git}.
%%  Also notice that {\tt [<args>]} follows {\tt <command>}; this means that commands will take arguments not listed here.
%%\item Read the description for each of the following commands: {\tt add}, {\tt checkout}, {\tt commit}, {\tt diff}, {\tt grep}, {\tt init}, {\tt log}, {\tt rm}, and {\tt status}.
%%  You will use these commands during this lab.
%%\item Confirm that commands do in fact take unlisted arguments.
%%  View the help for the {\tt init} command:\\
%%  \verb+$ git help init+\\
%%  As an alternative, try using the {\tt man} command:\\
%%  \verb+$ man git-init+\\
%%\end{enumerate}
%%

\section*{Productively Lost}

Being productively lost means to be trying things, breaking things and working
to develop and understanding.  You can read documentation for days about many
projects, but learn more in the first hour of actually working with the code.

In addition, many of you may never have built and run other people's code from
scratch, at least not code of significant size.  Applications don't pop out of
the ground ready to run, there is a process to making application deployable 
and you will learn about this process.

Part A will involve three main tasks:

\begin{enumerate}
\item Exploration 
\item Personal Documentation
\item Presentation and Summary
\end{enumerate}

\section*{Exploring}

The first phase will be exploring.  

\subsection*{The Community}

{\bf Find out all you can about your FOSS project and it's community}.  

\begin{itemize}
  \item Do they have a website?
  \item Where is their documentation?  
  \item How is their source code available?  
  \item Are there errors in the installation documents?
  \item How does the community communicate?
\end{itemize}

The community and the code are both crucial.  One of your team must join the
community via the community's selected communication tool.

\subsection*{The Code}

Basically, try to figure out how the code
does what it does.  Don't aim too high.  Since you're building a web-app, try
and change a behaviour on the main page.  Change a colour, modify the text
displayed.  Making little changes is how you can get your feet wet with
a project.  

Set a time with your group (lab time included, but you should meet at other
times too).  Work together, one computer, and try and understand the code.

{\bf At times you will feel lost and totally confused.}  The software you are
working with is 
complicated, you will not understand it all after an hour of playing with it.
If you get stuck trying to figure how something works, either ask a question of
your instructor or move on to some other feature.  I hope this experience will
serve well when faced with large codebases at your first job/internship.

Keep track of how much time you spent working together.

{\bf One aspect you must explore is the use of ``Plugins''.}  Both Ushahidi and
IPython support plugins as an easier way to make changes.  See if you can find
an existing plugin to install and try to understand how it works.  Ask questions, read documentation, etc.!

\section*{Personal Documentation}

While exploring, document what you learn.  Write down facts you figure out
about the community and the code.
Did you figure out how to pieces of code interact?  Great, draw a picture.
{\bf Remember, pictures are worth $2^{10}$ words.}

Your personal documentation should be in some kind of electronic document.
You may use a google doc or whatever, but it should be a document that anyone
can access and add to at any time.  Your personal documentation will be shared
with other groups working on the same project.

Your documentation {\bf must} contain the following:

\begin{itemize}
  \item A screenshot of the community's communication tool in action
  \item An architectural diagram of how the software works to the best of your
    knowledge
  \item Step-by-step instructions of how you reproduce one change you made to
    the software.
\end{itemize}

\section*{Presentation}

At the end of two weeks, you will give a 5-10 minute presentation on what
you've learned about your project to the class.  You must be able to show some
changes you've made to the code and what you've learned about the code base.

% Should the student's present a bit of a project proposal, where they will be going?

\section*{Marking}

25\% of your project mark (10 term marks) is dependent on Part A.  The
breakdown of marks for Part A is:

\begin{description}
\item[30\%] Presentation
\item[30\%] Personal Documentation about the community
\item[40\%] Code changes you've produced
\end{description}

\section{Submissions}

Submissions will be due at dates specified. Your documentation is due on
Monday Feb. 10.   Presentations will be done during the Lab Feb. 12th.

%%
%%\subsection*{Resources}
%%
%%\begin{footnotesize}
%%\begin{description}
%%\item[{[1]}] Ushahidi API document, {\tt http://wiki.ushahidi.com/doku.php?id=ushahidi\_api}
%%\item[{[2]}] Ushahidi Coding Style, {\tt
%%http://wiki.ushahidi.com/doku.php?id=coding\_standards}, broken link?  See
%%{\bf[3]}
%%\item[{[3]}] {\bf Google is {\it still} your friend}
%%\end{description}
%%\end{footnotesize}
%%
\end{document}

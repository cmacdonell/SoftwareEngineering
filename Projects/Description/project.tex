\documentclass[letterpaper]{article}

\usepackage{url}
\usepackage[letterpaper,top=1in,bottom=1in,right=1in,left=1.2in]{geometry}
\usepackage{lastpage}
\usepackage{fancyhdr}
\pagestyle{fancy}
\usepackage{newcent}
\usepackage{color}
\usepackage{keystroke}
\usepackage{ifthen}
\usepackage{graphicx}

\newcommand{\marginpoints}[1]{\marginpar{\vspace{0.3cm}\hspace{1.5cm}\Huge\sfrac{}{#1}}}
\newcommand{\important}[0]{\marginpar{\hspace{2mm}\footnotesize \textcolor{red}{IMPORTANT}}}

\newcommand{\course}{CMPT 395}
\newcommand{\activity}{Course Project}
\newcommand{\assigned}{today}
\newcommand{\due}{}
\newcommand{\duetime}{}
\newcommand{\weight}{40\% of final grade}

\newcommand{\horizrule}{\noindent\rule{\linewidth}{0.15mm}}

\lhead{\includegraphics[height=0.8cm]{../../Images/macewan.jpeg}}
\chead{\course{} \activity{}}
\rhead{Weight: \weight{}}
%\rhead{Due \due{} at \duetime{}}
\cfoot{Page \thepage\ of \pageref{LastPage}}

%\newenvironment{answer}{\begin{comment}}{\end{comment}}

\newenvironment{answer}
{
  \color{red}
}
{
  \vspace{3mm}
}

\title{\course{}: \activity{}}
\date{}

\begin{document}

\reversemarginpar

\begin{center}
  \Large{\activity{}}
\end{center}

\begin{center}
  \large{Objectives}
\end{center}

The goal of the project is to apply what you've learned (and will continue to
learn) about software development to work
on the planning and development of a real software project.
You will work in groups of three.

You may work on Ushahidi or IPython.  This course project may be different
than others you have done in the past because it will depend not only on the
final delivered product, but on the process you used to complete it, your
documentation and communication.  In particular, your ability to communicate
and the proper use of tools will impact your grade.

There are certainly going to be clarifications to the description below that
will be given in class or in the Blackboard discussion section.  The following
will convey the essence of the project.
\horizrule

\begin{center}
  \large{Instructions}
\end{center}

%%
%%
%%You will access {\tt git} using its command-line interface.
%%In this section, you will learn how to access its help system.
%%
%%\begin{enumerate}
%%\item Within a Terminal window, view {\tt git}'s most commonly used commands:\\
%%  \verb+$ git+
%%\item Look at the {\tt usage} line.
%%  Notice the small list of arguments that apply to {\tt git}.
%%  Also notice that {\tt [<args>]} follows {\tt <command>}; this means that commands will take arguments not listed here.
%%\item Read the description for each of the following commands: {\tt add}, {\tt checkout}, {\tt commit}, {\tt diff}, {\tt grep}, {\tt init}, {\tt log}, {\tt rm}, and {\tt status}.
%%  You will use these commands during this lab.
%%\item Confirm that commands do in fact take unlisted arguments.
%%  View the help for the {\tt init} command:\\
%%  \verb+$ git help init+\\
%%  As an alternative, try using the {\tt man} command:\\
%%  \verb+$ man git-init+\\
%%\end{enumerate}
%%

\section{Projects}

Your project will be split into three phases:

\begin{enumerate}
\item Exploration 
\item Development
\item Summary and Presentation
\end{enumerate}

The first phase will be exploration.  During this phase (about 2
weeks) you will learn what is possible with your software base.  You will then
meet with a ``client'' and discuss your project.   You should have some form
of a prototype (likely diagrams or mock-ups) of how the software will look and
work.  You will use github for version control.  

The second phase will involve designing and coding your implementation of your
project.  

After the development phase, you will present what you have done.  However,
this summary should be written from the frame of mind that the project is
continuing.  In this document you should describe what you have learned through
development.  Are certain features or requirements harder
than expected?  Were some requirements altogether infeasible?  Do you need to
adjust your requirements?

You will do two (2) presentations during the project.  The first will follow
your two week exploration and then a final project
presentation and demo of what you have working near the end of term.  The
presentations must be well-organized and professional in their delivery.

Different groups may select the same project.  In fact, it will be interesting
to compare and contrast the different approaches two separate groups may bring
to the same problem.


\vspace{0.2cm}
\begin{center}
\fbox{%
   \parbox{0.8\linewidth}{%
    \begin{center}
      {\bf ``Amount of Work'' Disclaimer}
    \end{center}
We will select projects from a list of suggestions that will be released to
you, or
you can suggest your own project.  Given that we are working together on a real
software application, I must acknowledge that the scale or work is a bit unclear, since
we are working on ``open'' problems.  If a project turns out to be too easy and
is completed quickly, I will require you take on some additional tasks.
Conversely, if a project is extremely difficult, I may scale it back to a
portion of the original assignment.  This is where your feedback and reporting
will be crucial.  I acknowledge that this is
different than a typical course project, but I believe that achieves two
things:

\begin{enumerate}
\item It makes the projects more interesting
\item It more closely resembles actual real-world software development
\end{enumerate}

   }
}
\end{center}

\section*{Weekly Meetings}

You will meet weekly in person with your project client.  With your group, you
agree on a meeting time outside of class and lab time in which you can meet
with your client.  Your first meeting will occur the week of January 27th.

Cam is the client for Ushahidi and Greg Wilson is the client for IPython.

You are expected to make progress weekly (sometimes a little, sometimes
a lot), give brief updates and ask questions.  There will be other venues to
ask questions as well such as your project's communication forum.

In general, the less progress you have made, the more questions you should have.

\section*{Project Topics}

As mentioned, in this course you will work with either Ushahidi or IPython,
the following are basic descriptions.  But remember, the approach of this
course is to challenge you to explore and put your own creativity into your
project.  For this reason the following descriptions are not extremely
detailed.

\subsection*{IPython Turtles Integration}

The default Python distribution comes with a Turtles interface for giving
instructons for a turtle to draw lines.  The turtle can move straight as well
as turn.  The turtle can also support different colours.

I suggest that you search online for Python Turtle examples and play around with it.

You have seen the IPython notebook demoed briefly in class.  You will install
it and also familiarize yourself with it.

The goal of this project is to add support for the Turtle interface to IPython.

You will need to play around with IPython and Python Turtles to understand the two
and how they would integrate.  Additional information from Greg Wilson will be provided during weekly meetings.

\subsection*{Ushahidi Analytics}

Ushahidi supports maps for displaying images, but can be somewhat limited in
it's support for other forms of information display (graphs, stats, etc.).

There are three projects available for Ushahidi:

1)  Add graphs to the Ushahidi front end (web display).  You will have to
research javascript graphing support and see what is possible.

2)  Add a heatmap display to the mainpage.  This project will involve adapting
an existing plugin to work with the main display.

3)  If after the first two weeks of ``Exploration'' you come up with an
interesting project of your own, you may present it as a potential project.

\section*{Marking}

40\% of your final mark is dependent on this project.  Within the project
itself, the breakdown of the percentages is as follows:

\begin{description}
\item[25\%] 2-week Exploration
\item[30\%] Functionality implemented (quality, completeness, features and documentation)
\item[25\%] Final write-up and demo
\item[20\%] Team members evaluation
\end{description}

\section {Grading Criteria}
\begin{description}
%\subsection{Weekly Progress}
\item[Communication] With all software development, communication is key.  You
will be graded based upon your progress as well as your communication of that progress.
%\subsection{Functionality}
\item[Functionality] This grade will reflect the functionality of the implementation you
have developed over the course of the development phase.
%\subsection{Demo}
\item[Demo]
As (we will see) it says in the text ``Scrum and XP...'' having something
working is not the same as having it demo-able.  A demo must show some
functionality that is part of the requirements or features of the system.
Don't just slap up whatever you have that day.  Think ahead.
%\subsection{Documentation and Code/Patch Quality}
\item[Documentation and Code/Patch Quality]
As with the first assignment, you will be graded on the code you wrote to
implement your project.  You will submit the code for your project at the end
of the development phase.  Code should well-documented and be contained in
git commits on your project git account.
\item[Progress]
You must make regular progress in your project.  Your instructor will monitor
git and meet with you regularly.
\end{description}

\section{Final Write-Up}

At the end of the term, a professional final write-up is expected that
describes what your project accomplished, as well as a continued development
plan.  Details will follow as to what the write-up should contain.  However,
you should not simply wait until the end of the term and write in retrospect.
You should make regular notes about what you are working on and time spent on
various tasks in your wiki or whatever.
Updates or modification to requirements should also be suggested in your final
write-up.

\section{Submissions}

Submissions will be due at dates specified.  You will meet with your client
during the week of January 27th. 

%%
%%\subsection*{Resources}
%%
%%\begin{footnotesize}
%%\begin{description}
%%\item[{[1]}] Ushahidi API document, {\tt http://wiki.ushahidi.com/doku.php?id=ushahidi\_api}
%%\item[{[2]}] Ushahidi Coding Style, {\tt
%%http://wiki.ushahidi.com/doku.php?id=coding\_standards}, broken link?  See
%%{\bf[3]}
%%\item[{[3]}] {\bf Google is {\it still} your friend}
%%\end{description}
%%\end{footnotesize}
%%
\end{document}
